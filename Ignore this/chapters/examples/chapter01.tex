\chapter{Main game play}
\label{ch:gameplay}

\section{Feeding}
\label{sec:gameplay:Feeding}

Feeding has two phases one for herbivores and one for carnivores.\newline
Omnivores can partcipate in both phases.\newline
\newline
1. Phase\newline
\newline
Herbivores/Omnivores will feed from a pool.\newline
(1d100+\ac{S})* \ac{FM}* square(\ac{T})=\ac{F}.\newline
\newline
2. Phase\newline
\newline
Carnivores/Omnivores will feed by attacking other animals\newline
(\ac{D}*square(\ac{T})*\ac{S}-(\ac{A}*square(\ac{T})*\ac{S})=\ac{F}.\newline
\newline
If a species has no population it becomes extinct.

\section{Social}
\label{sec:gameplay:Social}

This phase determines how they interact with other species in their territory:
\begin{itemize}
	\item Aggressive
	\item Cooperative
	\item Neutral 
	\item Defensive
	\item Hostile
	\item Special
\end{itemize}
Aggressive stance increases the amount of territory you control, but reduces the efficiency of your feeding\newline +10\% \ac{TM}, -10\% \ac{FM}\newline\newline
Cooperative decreases the amount of territory you control, but increases the efficiency of your feeding\newline -15\% \ac{TM}, 15\% \ac{FM} (my idea also add "`trade"' deals).\newline\newline
Neutral\newline You gain no modifiers\newline\newline
Defensive\newline -10\% \ac{TM} + 2 \ac{D}\newline\newline

\begin{table}[hp]
	\centering
		\begin{tabular}{|l|l|l|}
		\hline
		Roll & Description & Effect\\
		\hline
		0 & blah & +10\% \ac{TM}\\
		\hline	
		1 & xxx & +5\% \ac{FM}\\
		\hline
		2 & aasd & \ac{TM}\\
		\hline
		3 & asd & Roll species event\\  
		\hline
		\end{tabular}
	\caption{Events}
	\label{tab:Events}
\end{table}

\begin{table}[h]
	\centering
		\begin{tabular}{|l|l|l|}
		\hline
		Roll & Description & Effect\\
		\hline
		0 & asd & symbiosis\\
		\hline	
		1 & asd & test\\
		\hline
		\end{tabular}
	\caption{SpecialEvent}
	\label{tab:SpecialEvent}
\end{table}

\section{Evolution}
\label{sec:gameplay:Evolution}

Players can evolve traits, Each trait cost EVO points, you may also upgrade traits.
There are many traits, so will need to ask GM for trait cost if it is not listed here.
You can spend evo points equal to a skill to level it up, so to level up explore from 5 to 6, would cost 5 evo points.
Species generally need a trait for reproducing/breathing/eating/moving.
\newline
You gain 1 EVO points each time your species loses population, from natural selection.
Your species also gains 5 Evo points each time you reach a milestone population, this includes your starting pop.\newline10,20,50,100,200,500,1000,2000,5000…
You gain 5 EVO points each time you can fully feed your species, otherwise gain 3 EVO points.
\newline
Your first 3 non-innate traits have no additional cost, after that you must spend an additional 5 evo points per trait, this cost increases by 5 for each set of 3 traits you have (not including your first).
\newline
You can spend 10\% of your pops worth food to increase your intelligence by 1. Intelligence has milestones at 5/10/15/20/25… giving 5 evo points each time. You cannot buy intelligence with evo points


\section{Mating}
\label{sec:gameplay:Mating}
Mating equations can vary based on sexual organ types, and birthing methods. But a general equation will go as such:\newline
30\% of current population* pop growth multiplier\newline\newline
If food required is higher then food scavenged use the following:\newline
-30\% of current population*((food required-food scavenged)/food required)*pop growth multiplier.

\section{Event}
\label{sec:gameplay:Event}
These represent big changes to the world, these can positively or negatively affect the amount of food in pools, events can cause new species to emerge, and open new areas of the map.
\begin{table}[h]
	\centering
		\begin{tabular}{|l|l|l|}
		\hline
		Roll & Description & Effect\\
		\hline
		0 & asd & test\\
		\hline	
		1 & asd & test\\
		\hline
		\end{tabular}
	\caption{MajorEvent}
	\label{tab:MajorEvent}
\end{table}